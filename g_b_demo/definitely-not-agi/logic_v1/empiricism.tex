\documentclass{article}
\usepackage[utf8]{inputenc}
\usepackage{amsmath}
\usepackage{hyperref}
\usepackage{amssymb}
\usepackage{geometry}
\geometry{margin=1in}

\title{Intersubjectivity and the Philosophy of Mathematics and Science}
\author{Brian Searls\\\href{https://github.com/briansrls}{https://github.com/briansrls}}
\date{\today}

\begin{document}

\maketitle

\begin{abstract}
This paper presents an intersubjective perspective on the philosophy of mathematics and empirical science. It argues that mathematics is best understood not as an objectively existing realm nor as a subjective personal construct, but rather as an intersubjectively consistent logical framework developed through collective human agreement. By clarifying the epistemological boundaries of mathematics and science, we provide a nuanced philosophical approach that resolves longstanding debates concerning objectivity, subjectivity, and the limits of empiricism.
\end{abstract}

\section{Introduction}

The philosophy of mathematics and science traditionally oscillates between objectivist (Platonist) and subjectivist (constructivist) viewpoints. While objectivists argue mathematics reflects an independently existing logical structure, subjectivists claim mathematics arises purely from human cognition. Here, we present a third, nuanced view: mathematics as an \textit{intersubjective logical understanding}.

\section{Intersubjectivity Defined}

Intersubjectivity is distinct from objectivity and subjectivity in important ways:

\begin{itemize}
    \item \textbf{Objectivity} claims independent existence external to any observer.
    \item \textbf{Subjectivity} restricts knowledge to individual experience.
    \item \textbf{Intersubjectivity}, however, emerges from shared agreements among observers, enabling collective consistency and verification.
\end{itemize}

Intersubjective truths do not necessarily reflect objective reality; rather, they reflect stable agreements among human observers, allowing them to reliably structure, predict, and navigate their experiences.

\section{Mathematics as an Intersubjective Framework}

Mathematics exemplifies intersubjectivity because it:

\begin{enumerate}
    \item Is based on collective acceptance of axioms, logical rules, and definitions.
    \item Allows consistent, reproducible logical proofs independent of subjective opinion.
    \item Provides universally agreed-upon structures (numbers, functions, geometry), despite no absolute proof of their external existence.
\end{enumerate}

Mathematics does not necessitate external objectivity—it simply requires internal logical coherence agreed upon by multiple observers. Consequently, its effectiveness arises precisely from this shared structure rather than correspondence with objective reality.

\section{Implications for Empiricism}

Empirical science uses mathematics to approximate, describe, and predict physical phenomena. However, the inherently intersubjective nature of mathematics implies that:

\begin{itemize}
    \item Scientific theories remain approximations: mathematical models never perfectly match physical reality due to quantum uncertainty, measurement limitations, and idealizations.
    \item Empirical "truths" are best understood as reliable intersubjective agreements rather than absolute facts about objective reality.
\end{itemize}

This realization demands humility about scientific claims. Science excels at producing intersubjectively stable models—highly accurate, yet always approximate, descriptions of reality.

\section{Philosophical Consequences}

Viewing mathematics and science as fundamentally intersubjective reshapes our epistemological stance:

\begin{itemize}
    \item It resolves debates between realism and anti-realism by positioning mathematics and science as intersubjective tools rather than absolute truths.
    \item It emphasizes intellectual humility, highlighting limits inherent in empirical and logical systems due to their dependence on collective agreement.
    \item It grounds philosophical inquiry in pragmatic clarity, avoiding unprovable claims about objective existence while respecting mathematics and science's immense pragmatic value.
\end{itemize}

\section{Conclusion}

Recognizing mathematics and science as intersubjective frameworks reconciles longstanding philosophical tensions between realism and constructivism. It reframes empirical truths as reliable agreements rather than absolute certainties, emphasizing the practical and epistemological humility necessary for genuine intellectual progress.

\end{document}
