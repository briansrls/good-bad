\documentclass[12pt]{article}
\usepackage[utf8]{inputenc}
\usepackage{amsmath,amssymb}
\usepackage{geometry}
\usepackage{setspace}
\usepackage{hyperref}
\geometry{margin=1in}
\onehalfspacing

\title{A Short Logical Path to Modern Math:\\
From Boolean Algebra to Calculus}
\author{Brian Searls\\\href{https://github.com/briansrls}{https://github.com/briansrls}}
\date{\today}

\begin{document}

\maketitle

\begin{abstract}
This document provides a concise tour of some core mathematical ideas,
ranging from the basic laws of logic (Boolean algebra) through arithmetic,
algebra, and the essential elements of a first course in calculus.
At each stage, we adopt the viewpoint that logic serves as a system for
labeling statements as ``good'' (true, acceptable) or ``bad'' (false,
unacceptable), then show how the familiar structures of arithmetic and
calculus can be built on top of that foundation.
\end{abstract}

\tableofcontents

\section{Introduction}
Classical logic is often framed as assigning statements the values
\textbf{true} or \textbf{false}. Interpreted intuitively, we can call
true statements ``\emph{good}'' (acceptable) and false statements
``\emph{bad}'' (unacceptable). This simple labeling framework can
be extended step by step to create the rich structure we call
\emph{modern mathematics}.

\section{Boolean Algebra: The Rules of ``Good/Bad'' Reasoning}

\subsection{Basic Operations}
\label{sec:boolean-ops}
Boolean algebra captures the core logical connectives using symbols
like:
\[
\begin{aligned}
P \wedge Q &: \text{AND (both must be good)} \\
P \vee Q &: \text{OR (at least one is good)} \\
\lnot P &: \text{NOT (the statement $P$ is bad)} \\
P \to Q &: \text{IMPLIES (if $P$ is good, $Q$ must be good)}
\end{aligned}
\]
Within Boolean algebra, each proposition $P$ is assigned a truth value
(1 for good, 0 for bad). From here we can handle everyday logic or
mathematical statements in a formal, consistent manner.

\subsection{Consistency Principle}
The core principle remains the \textbf{Law of Noncontradiction}: no
statement can be both good (1) and bad (0) at the same time in the
same context. Equivalently, $P \wedge \lnot P$ is always labeled 0.
Everything else follows from the rule that our truth assignments must
be consistent.

\section{From Logic to Arithmetic}
\label{sec:logic-arith}

\subsection{Counting: Natural Numbers}
Classical \emph{logicism} (Frege, Russell, etc.) showed how to build
the \emph{natural numbers} out of pure logic. Practically:
\begin{itemize}
  \item Counting identical items (e.g., grains of rice) can be seen as
  summing up many separate ``good'' instances of ``this is a grain.''
  \item The \emph{natural numbers} $\{0,1,2,3,\dots\}$ arise by repeating
  ``another true item'' and labeling the total.
\end{itemize}

\subsection{Negative Numbers and Zero}
Numbers extend to the \emph{integers} $\{\dots,-2,-1,0,1,2,\dots\}$ when
we allow for \textbf{debts of truth} or deficits. If we owe 3 grains,
we say we have $-3$. The integer $0$ labels ``no items, no debt''
(a neutral starting point).

\subsection{Rational and Real Numbers}
\begin{itemize}
  \item \textbf{Rationals} (fractions) let us describe \emph{partial
  truths}: we treat one whole unit as splittable into equal parts. 
  Two halves (each labeled $0.5$) make a whole by consensus.
  \item \textbf{Real numbers} fill in all the \emph{gaps} between
  fractions, capturing limits of sequences or endless decimal expansions.
  Logically, reals can be built via \emph{Cauchy sequences} or \emph{Dedekind
  cuts}, each of which is a consistent labeling scheme for infinite
  approximations.
\end{itemize}
Thus, from a logic-first perspective, real numbers emerge from rules
that let us consistently talk about infinitely fine subdivisions and
limits, ensuring no internal contradictions.

\section{Algebra: Rearranging and Combining Equalities}
\label{sec:algebra}

\subsection{Basic Equations}
Algebra deals with statements of the form ``$x + 3 = 10$'' and sees
them as \emph{questions} about $x$. If $x + 3 = 10$ is good (true),
then we can systematically manipulate that statement to find $x = 7$.
\emph{Why is it valid?} Because we have rules of logic that ensure:
\[
\text{``If (x + 3 = 10) is good, then (x = 7) must also be good''}
\]
once we accept standard arithmetic rules. This is exactly the meaning
of inference in a consistent logical framework.

\subsection{Polynomials and Factorization}
We can build polynomials, like $x^2 - 5x + 6 = 0$, and ask if there's
a good assignment of $x$ that satisfies it. The factoring step
\[
x^2 - 5x + 6 \;=\; (x-2)(x-3)
\]
follows from arithmetic laws, ensuring we rewrite the same expression
without contradicting any steps. So $(x=2)$ or $(x=3)$ each make the
original equation true (good).

\section{Foundations of Calculus: Limits, Derivatives, and Integrals}
\label{sec:calculus}

\subsection{Limits as Logical Approximations}
A \textbf{limit} describes what happens to a function $f(x)$ as $x$
approaches some value $a$. In logic terms, it's a statement about
\emph{consistency} among infinitely many approximate claims:
\[
\forall \varepsilon>0,\; \exists \delta>0:
   \text{ if } 0 < |x - a| < \delta \text{ then } |f(x) - L| < \varepsilon.
\]
We label a limit ``good'' if it satisfies that statement for every
$\varepsilon>0$. If any contradiction arises (i.e., we find a case
that violates the definition), the limit claim is false.

\subsection{Derivatives: Rates of Change}
The \textbf{derivative} $f'(a)$ is a limit describing the slope of
$f(x)$ at $x=a$. Logically, it's a \emph{compound} statement: 
\[
f'(a) = \lim_{h \to 0} \frac{f(a+h) - f(a)}{h}.
\]
If that limit is consistent, we label it as the derivative. If it
causes contradictory or undefined behavior (like division by zero
that can't be resolved via a limit), we say the derivative does not
exist there.

\subsection{Integrals: Summing Infinitesimal Pieces}
The \textbf{integral} $\int_a^b f(x)\,dx$ can be seen as an infinite
sum of $f(x)$ values over tiny slices of $x$. We often define it as
the limit of Riemann sums:
\[
\lim_{n\to\infty} \sum_{k=1}^{n} f(x_k)\,\Delta x,
\]
with $\Delta x = \frac{b-a}{n}$. Each term $f(x_k)\,\Delta x$ is a
\emph{partial truth}, i.e.\ the area of a thin rectangle. If all these
partial sums fit together consistently as $n$ grows, then the integral
is ``good'' and well-defined. 

\subsection{Fundamental Theorem of Calculus}
The \textbf{Fundamental Theorem of Calculus} (FTC) states that
integration and differentiation are \emph{inverse} processes, under
certain mild conditions. Symbolically:
\[
\frac{d}{dx} \left( \int_{a}^{x} f(t)\,dt \right) = f(x).
\]
This can be seen as a major logical link between two structures
(derivatives and integrals). Once again, it relies on consistent
definitions of limits and sums. If any contradiction were introduced
(e.g., summing with no coherent limit), the theorem wouldn't hold. 
But the standard rules keep everything consistent, so the statement
remains ``good'' in that framework.

\section{Wrap-Up: The Logical Backbone of Modern Math}

\subsection{From Good/Bad Labels to Broad Theories}
We started with the idea that logic classifies statements as true/good
or false/bad. From there, we built up:
\begin{enumerate}
  \item \textbf{Boolean Algebra} for combining statements consistently.
  \item \textbf{Arithmetic} from simple counting of identical items
  through negative numbers (debts of truth) and fractions (partial truths).
  \item \textbf{Algebra} for rearranging good statements to solve
  equations.
  \item \textbf{Calculus} for studying \emph{continuous} behavior:
  limits, derivatives, and integrals—all hinged on consistent rules that
  define infinite processes without contradiction.
\end{enumerate}

\subsection{Why This View Helps}
Seeing math as an extension of logic clarifies why consistency is so
important. If we accidentally allow a contradiction, we could label
everything as ``true,'' making math meaningless. Instead, each step must
respect the rules (like the Law of Noncontradiction, definitions of limit,
and so on). The result is a robust structure that powers science and
engineering.

\begin{thebibliography}{9}

\bibitem{Boole}
G. Boole.
\textit{An Investigation of the Laws of Thought}.
Walton and Maberly, 1854.

\bibitem{Frege}
G. Frege.
\textit{The Foundations of Arithmetic}.
1884. (Various editions, e.g.\ Northwestern University Press.)

\bibitem{RussellWhitehead}
B. Russell and A. N. Whitehead.
\textit{Principia Mathematica}.
Cambridge University Press, 1910--1913.

\bibitem{Godel}
K. G\"odel.
\textit{On Formally Undecidable Propositions of Principia Mathematica 
and Related Systems I}.
1931. (English translations in many anthologies.)

\end{thebibliography}

\end{document}
