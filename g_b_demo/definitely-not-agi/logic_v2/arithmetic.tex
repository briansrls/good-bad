\documentclass[11pt]{article}
\usepackage[margin=1in]{geometry}
\usepackage{hyperref}
\usepackage{amsmath,amssymb,amsthm,amsfonts,stmaryrd}

\title{A Basic Arithmetic in Good/Bad Logic}
\author{Brian Searls\\\href{https://github.com/briansrls}{https://github.com/briansrls}}
\date{\today}

\begin{document}
\maketitle

\section{Introduction}

In prior work, we introduced a two-valued logic (G--B logic) where each proposition 
is assigned a pair of real values
\[
   \mathcal{V}(\varphi) \;=\; \bigl(G(\varphi),\,B(\varphi)\bigr),
\]
with $G(\varphi), B(\varphi) \in [0,1]$, reflecting the \emph{goodness} and \emph{badness} of the statement. 
Here, we extend that approach to a first-order setting with basic arithmetic over the natural numbers. 
Our goal is to show, at a high level, how G--B valuations can carry over to standard arithmetic 
concepts such as $0,1,+,\times$ and quantifiers $\forall,\exists$.

\section{Syntax of the Theory}

\subsection{Language of Arithmetic}

We use a standard first-order language $\mathcal{L}_A$ with:
\begin{itemize}
    \item A constant symbol $0$ (zero).
    \item A unary function symbol $S(\cdot)$ (successor).
    \item Binary function symbols $+,\times$ (addition, multiplication).
    \item A binary relation symbol $=$ (equality).
    \item Logical connectives: $\land,\lor,\lnot,\to$.
    \item Quantifiers: $\forall,\exists$.
\end{itemize}

Terms are defined as usual: 
\[
  t ::= 0 \;\mid\; S(t) \;\mid\; t + t \;\mid\; t \times t \;\mid\; x_i,
\]
where $x_i$ are variables. Formulas are built from atomic formulas like $t_1 = t_2$ using 
$\land,\lor,\lnot,\to,\forall,\exists$.

\section{Semantics with G--B Valuations}

We assume a domain $\mathcal{D}$ (for concreteness, $\mathcal{D} = \mathbb{N}$). 
Each function symbol is interpreted in the usual way over $\mathbb{N}$:
\[
   0^\mathcal{D} = 0,\quad
   S^\mathcal{D}(n) = n+1,\quad
   +^\mathcal{D} = \text{standard addition},\quad
   \times^\mathcal{D} = \text{standard multiplication}.
\]
The crux is how to \emph{assign good/bad values to formulas}. Let $\varphi$ be a formula. 
We define $\mathcal{V}(\varphi) = (G(\varphi), B(\varphi))$ as follows:

\subsection{Base Case: Atomic Formulas}

An atomic formula has the form $t_1 = t_2$, where $t_1,t_2$ are terms. 
Compute their standard interpretation in $\mathbb{N}$:
\[
  \llbracket t_i \rrbracket \in \mathbb{N}.
\]
Then define a \emph{goodness} measure $G(t_1 = t_2)$ and \emph{badness} measure $B(t_1 = t_2)$ 
based on whether they coincide numerically. For a simple version:
\[
  G(t_1 = t_2) = 
     \begin{cases}
       1, & \text{if } \llbracket t_1 \rrbracket = \llbracket t_2 \rrbracket, \\
       0, & \text{otherwise},
     \end{cases}
  \quad\text{and}\quad
  B(t_1 = t_2) = 
     \begin{cases}
       0, & \text{if } \llbracket t_1 \rrbracket = \llbracket t_2 \rrbracket, \\
       1, & \text{otherwise}.
     \end{cases}
\]
This \emph{mirrors classical equality} (true/false) but in G--B form. 
If you want partial matches (e.g., approximate arithmetic), you could define 
$G$ and $B$ differently (for instance, using $| \llbracket t_1 \rrbracket - \llbracket t_2 \rrbracket |$).

\subsection{Compound Formulas (No Quantifiers)}

Use the same definitions as in G--B propositional logic:
\[
  \mathcal{V}(\lnot \varphi) = (B(\varphi),\, G(\varphi)),
\]
\[
  \mathcal{V}(\varphi \land \psi)
    = \bigl(\min\{G(\varphi),G(\psi)\},\,\max\{B(\varphi),B(\psi)\}\bigr),
\]
\[
  \mathcal{V}(\varphi \lor \psi)
    = \bigl(\max\{G(\varphi),G(\psi)\},\,\min\{B(\varphi),B(\psi)\}\bigr),
\]
\[
  \mathcal{V}(\varphi \to \psi)
    = \mathcal{V}(\lnot \varphi \;\lor\; \psi).
\]

\subsection{Quantifiers}

For a formula $\forall x\, \varphi(x)$, we want to aggregate the G--B values of all 
instances $\varphi(n)$ for $n \in \mathbb{N}$. A simple aggregator is:
\[
  G(\forall x\, \varphi(x)) \;=\; 
   \min_{n \in \mathbb{N}} \; G(\varphi(n)),
  \quad
  B(\forall x\, \varphi(x)) \;=\;
   \max_{n \in \mathbb{N}} \; B(\varphi(n)).
\]
That is, the statement ``$\varphi(x)$ holds for all $x$'' is \emph{fully good} 
only if $G(\varphi(n))$ is high for \emph{every} $n$, 
and it is \emph{somewhat bad} if $B(\varphi(n))$ is high for \emph{any} $n$. 

Similarly, for $\exists x\, \varphi(x)$, a natural aggregator would be:
\[
  G(\exists x\, \varphi(x)) \;=\; 
   \max_{n \in \mathbb{N}} \; G(\varphi(n)),
  \quad
  B(\exists x\, \varphi(x)) \;=\;
   \min_{n \in \mathbb{N}} \; B(\varphi(n)).
\]
We deem an existential statement good if \emph{some} instance has high goodness, 
and we consider it bad if \emph{all} instances have high badness. 
(You can choose alternative aggregation schemes depending on philosophical or practical needs.)

\section{Arithmetic Axioms in G--B Form}

We can incorporate standard Peano-like axioms, now interpreted in G--B style. For example:

\begin{enumerate}
  \item \textbf{Zero is not the successor of any number:} 
  \[
    \forall x\, (S(x) = 0) \quad \text{(should be ``bad'')} 
  \]
  implies $G(\forall x\, S(x)=0)$ is $0$ in classical arithmetic, 
  $B(\forall x\, S(x)=0)$ is $1$, reflecting that the statement is false for all $x$.

  \item \textbf{Successor injectivity:} 
  \[
    \forall x \forall y\, \bigl( S(x)=S(y) \to x=y \bigr).
  \]
  Under the aggregator definitions, $G$ of this formula depends on 
  whether the statement holds for all $x,y$ in $\mathbb{N}$.

  \item \textbf{Commutativity of $+$:}
  \[
    \forall x \forall y\, (x + y = y + x).
  \]
  If classical arithmetic is correct, we expect $G(\forall x\forall y\, x+y=y+x)=1$ 
  and $B(\forall x\forall y\, x+y=y+x)=0$, 
  because it indeed holds for \emph{every} pair $x,y$ in $\mathbb{N}$.

  \item \textbf{Induction Schema:}
  Typically, we have:
  \[
    \bigl[\varphi(0) \land \forall x\,(\varphi(x)\to \varphi(S(x)))\bigr]
    \;\to\; \forall x\, \varphi(x).
  \]
  In G--B logic, each piece is assigned a $(G,B)$ pair, 
  aggregated via $\land,\to,\forall$ as above. 
  If induction is \emph{true} in the classical sense, we expect it to yield 
  high $G$-values for all instances.
\end{enumerate}

\noindent
Thus, classical arithmetic emerges as the case where these axioms 
consistently evaluate to $(1,0)$, i.e.\ fully good and not bad at all.

\section{Discussion and Future Directions}

This basic arithmetic extension demonstrates how first-order statements 
involving $0,S,+,\times$ can be assigned G--B values. The aggregator approach 
to quantifiers---using $\min$ over $G$ and $\max$ over $B$ for universal, 
and $\max$ over $G$, $\min$ over $B$ for existential---mimics typical 
\emph{fuzzy} or \emph{t-norm} strategies, but preserves the possibility 
of partial contradictions. 

\begin{itemize}
    \item \textbf{Approximate Arithmetic:} If we allow near-equalities, 
    $G(t_1=t_2)$ might reflect how close $t_1, t_2$ are numerically, 
    enabling a more tolerant system for real-world arithmetic or 
    measurement-based statements.
    \item \textbf{Lifting Classical Theorems:} Many standard results 
    (e.g.\ the sum of angles in a triangle) could be re-expressed 
    with G--B valuations, verifying that they still come out 
    ``fully good'' $(1,0)$ in purely classical domains.
    \item \textbf{Implementation in AGI:} An AGI might treat 
    these valuations as part of a flexible knowledge base, 
    so that even if some arithmetic statements become contradictory 
    (due to noise or partial knowledge), the system does not explode 
    but updates $G,B$ accordingly.
\end{itemize}

\section{Conclusion}

By combining G--B logic semantics with the standard language of arithmetic, 
we obtain a system where each formula over $\mathbb{N}$ (with $0,S,+,\times$) 
can be assigned \emph{two} real values in $[0,1]$. The universal and existential 
quantifiers are handled via aggregator functions (e.g.\ $\min,\max$), 
extending the paraconsistent and multi-valued features of G--B logic 
to statements about numbers. 

Future work may explore more sophisticated assignments of $G,B$ to atomic 
formulas (beyond strict equality), or investigate alternative aggregator 
rules for quantifiers. Nonetheless, this draft demonstrates a feasible path 
to unify standard arithmetic reasoning with a multi-valued, paraconsistent 
framework that can better reflect the complexity encountered in real AI 
applications.

\end{document}