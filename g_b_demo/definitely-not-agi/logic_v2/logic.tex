\documentclass[11pt]{article}

\usepackage{amsmath,amssymb,amsfonts}
\usepackage{geometry}
\usepackage{graphicx}
\usepackage{hyperref}
\usepackage{tikz}
\usepackage{lipsum}

\geometry{
  a4paper,
  left=25mm,
  right=25mm,
  top=25mm,
  bottom=25mm
}

\title{A Multi-Valued Logic of Good/Bad for AGI Applications}
\author{Brian Searls\\\href{https://github.com/briansrls}{https://github.com/briansrls}}
\date{\today}

\begin{document}

\maketitle

\begin{abstract}
We present a new logical framework that treats every proposition as carrying both 
\emph{goodness} and \emph{badness} values, allowing partial overlap and thus going 
beyond the standard true/false paradigm. We define a semantics in which each statement
\(p\) is evaluated as a pair \((G(p), B(p))\), with \(G, B \in [0,1]\). This system 
is capable of tolerating internal contradictions (i.e., a statement can be partly 
good and partly bad) while still maintaining a coherent structure for inference. We 
propose that this good/bad logic is especially relevant to artificial general intelligence (AGI), 
where action selection and knowledge representation must often consider trade-offs, 
ethical dimensions, and uncertainty all at once. We show how classical logic emerges 
as a special case when \(G + B = 1\). The aim is to create a non-explosive, 
multi-dimensional logic that can handle real-world complexity and paradox, providing a 
potential substrate for next-generation reasoning systems in AGI.
\end{abstract}

\section{Introduction}

Artificial General Intelligence (AGI) research explores creating machines that can learn 
and reason across diverse tasks with minimal re-engineering. Traditional logical systems 
for knowledge representation---rooted in classical logic with bivalent true/false 
assignments---often struggle with ambiguity, ethical trade-offs, and paradox. 
Fuzzy and probabilistic logics make partial strides, but they still tend to unify all 
uncertainty and partial membership into a single scalar parameter.

In this paper, we present an alternative approach: \emph{a multi-valued logical system} in which 
each proposition \(p\) is assigned two real values in \([0,1]\):
\[
   \mathcal{V}(p) = \bigl(G(p), B(p)\bigr),
\]
where \(G(p)\) represents the ``goodness'' (positive survival alignment) of \(p\) 
and \(B(p)\) represents its ``badness'' (negative survival alignment).
Crucially, \(G\) and \(B\) can overlap and do not necessarily sum to 1, 
yielding a paraconsistent-like framework that tolerates contradictions. 

We argue that such an approach is highly relevant to AGI, where managing internal contradictions 
and partial ethical judgments is essential. By permitting partial moral evaluations and 
\emph{non-explosive} contradictions, we lay the foundation for more robust, context-aware, 
and ethically informed reasoning within autonomous systems.

\section{Background and Motivation}

\subsection{Shortcomings of Classical Logic}

Classical logic provides powerful results such as completeness and soundness, 
but it imposes strict bivalence: every well-formed proposition is either true or false. 
In real-world contexts involving ethical dilemmas, uncertain data, or contradictory signals, 
this rigidity can lead to undesirable outcomes. A single contradiction often entails 
\emph{ex contradictione sequitur quodlibet} (explosive inference), trivializing the system.

\subsection{Fuzzy Logics}

Fuzzy logic~\cite{zadeh1965fuzzy} addresses some limitations of classical logic by allowing degrees 
of truth in \([0,1]\). This is valuable for modeling graded notions like ``tall'' or 
``heavy.'' However, standard fuzzy systems unify \emph{all} partial membership into a single 
scalar parameter, without explicitly capturing situations where something can be 
beneficial and harmful simultaneously, each to some degree.

\subsection{Why Good/Bad Dimensions?}

In human cognition, concepts or actions can be both good and bad in different respects. 
For instance, alcohol consumption might be ``relaxing'' (good) but also ``harmful to health'' 
(bad). Instead of forcing a single net result, we propose a logic where each proposition 
maintains its own \emph{goodness} and \emph{badness} components, with partial overlap allowed. 
This yields:

\begin{itemize}
    \item \textbf{Tolerance for Contradiction:} Contradictions do not cause the entire logic to explode.
    \item \textbf{Ethical Evaluations:} The additional dimension can incorporate or reflect 
    moral/ethical values for AGI, helping guide action selection.
    \item \textbf{Contextual Nuance:} A proposition may score moderately high in both 
    good and bad aspects, which can be critical for complex real-world trade-offs.
\end{itemize}

\section{Syntax}

We adopt a standard propositional language:
\begin{itemize}
    \item A countable set of propositional variables \(\{p,q,r,\dots\}\).
    \item Logical connectives: \(\land\) (AND), \(\lor\) (OR), \(\lnot\) (NOT), and \(\to\) (IMPLIES).
    \item Parentheses for grouping.
\end{itemize}

The system can be extended to first-order logic, but we focus here on the propositional level 
to define the core semantics.

\section{Semantics}

\subsection{Valuations}

Each proposition \(p\) is assigned an ordered pair
\[
   \mathcal{V}(p) \;=\; \bigl(G(p),\,B(p)\bigr),
\]
where \(G(p), B(p) \in [0,1]\). We do \emph{not} force \(G(p) + B(p) \le 1\); full overlap is allowed. 
If one prefers to restrict overlap, the constraint \(G(p) + B(p) \le 1\) can be imposed as an additional axiom.

\subsection{Connectives}

We define the semantics of the connectives as follows; these are one set of design choices and 
may be adapted to fit other use-cases:

\paragraph{Negation \(\lnot p\)}
\[
   \mathcal{V}(\lnot p) \;=\; \bigl(B(p),\,G(p)\bigr).
\]
Negating a proposition flips its goodness and badness components.

\paragraph{Conjunction \(p \land q\)}
\[
   \mathcal{V}(p \land q) \;=\;
   \bigl(\min\{G(p), G(q)\},\;\max\{B(p), B(q)\}\bigr).
\]
Intuitively, for a conjunction to be ``good,'' both conjuncts must be sufficiently good, 
hence the minimum. Any badness from either part potentially carries over, giving the maximum.

\paragraph{Disjunction \(p \lor q\)}
\[
   \mathcal{V}(p \lor q) \;=\;
   \bigl(\max\{G(p), G(q)\},\;\min\{B(p), B(q)\}\bigr).
\]
If either component is good, it elevates the disjunction's goodness. If one has low badness, 
that helps constrain the overall badness.

\paragraph{Implication \(p \to q\)}
A common approach is to define 
\[
    p \to q \;\equiv\; \lnot p \;\lor\; q,
\]
then apply the semantics of \(\lnot\) and \(\lor\). Concretely:
\[
   \mathcal{V}(p \to q) 
     = \max\bigl(\mathcal{V}(\lnot p), \,\mathcal{V}(q)\bigr)
     = \bigl(\max\{B(p), G(q)\},\,\min\{G(p), B(q)\}\bigr).
\]
Alternate definitions are possible, depending on the desired behavior for implication.

\section{Axioms and Inference}

\subsection{Axiom Schemas}

This logic is grounded in the following principles:

\begin{itemize}
    \item \textbf{Non-Explosion (Paraconsistency):} We do not enforce \((p \land \lnot p) \to q\) 
    for arbitrary \(q\). A contradiction in some statements does \emph{not} trivialize the entire system.
    \item \textbf{Classical Tautologies under Extremes:} If \(G(p)=1\) and \(B(p)=0\), we recover 
    the classical notion of ``fully true,'' and classical tautologies behave accordingly.
    \item \textbf{Excluded Middle Not Guaranteed:} We do not require \(p \lor \lnot p\) to be 
    \((1,0)\) in all cases; partial overlap of \(p\) and \(\lnot p\) is allowed.
\end{itemize}

\subsection{Inference Rules}

We can adapt common rules from natural deduction or Hilbert systems:

\paragraph{Modus Ponens (Threshold-Based):}
\[
  \frac{p,\quad p \to q}{q}
\]
In a multi-valued logic, we may require thresholds:
if \(G(p)\) and \(G(p \to q)\) both exceed some \(\alpha\), we conclude \(G(q)\) exceeds some 
\(\alpha'\). This preserves a non-binary style of reasoning.

\paragraph{Intro/Elim for \(\land\) and \(\lor\):}
We adapt standard introduction/elimination rules for conjunctions and disjunctions, 
but interpret them via the G--B valuations to control how partial truths and partial 
badness propagate.

\paragraph{Monotonicity:}
If \(\mathcal{V}(p)=(g_p,b_p)\) and \(\mathcal{V}(q)=(g_q,b_q)\) with 
\(g_p \ge g_q\) and \(b_p \le b_q\), then we can consider \(p\) 
``at least as good (and not more bad)'' as \(q\). This allows certain 
substitutions or inferences, depending on additional rules about partial ordering.

\section{Recovering Classical Logic and Other Systems}

\subsection{Classical Logic as a Subcase}

If for every proposition \(p\) we force
\[
  G(p) \in \{0,1\}, \quad B(p) \in \{0,1\}, \quad 
  \text{and}\quad G(p) + B(p) = 1,
\]
we regain bivalent truth: \((1,0)\) is ``true,'' \((0,1)\) is ``false.'' 
Classical tautologies and contradictions then behave identically to classical logic. 
Thus, our multi-valued system \emph{extends} classical logic rather than discarding it.

\subsection{Comparison to Fuzzy Logic}

If we set \(B(p)=1-G(p)\) strictly for all \(p\), then we effectively return to 
standard one-dimensional fuzzy logic. Our system, however, is more general because 
it decouples the goodness and badness parameters, allowing partial overlap to handle 
cases where statements are simultaneously somewhat good and somewhat bad.

\subsection{Paraconsistency}

By design, the logic is paraconsistent: a single contradiction 
\((p \land \lnot p)\) does not entail universal derivability. This is crucial for 
real-world or AGI settings where contradictory evidence is common, yet we want 
the system to remain useful.

\section{AGI Applications}

\subsection{Knowledge Representation}

An AGI could store statements with valuations 
\(\mathcal{V}(p)=(G(p), B(p))\). For instance:
\begin{quote}
  ``Taking Action A reduces short-term stress'' \(\rightarrow (G(A)=0.8, B(A)=0.4)\).
\end{quote}
The partial badness might reflect negative side effects. Rather than collapsing to a single 
score, the system keeps both dimensions intact.

\subsection{Ethical Decision-Making}

Because the system explicitly encodes moral or survival relevance, it can be naturally extended 
into a decision process. If we have multiple candidate actions \(\{A_i\}\) with respective 
valuations \(\mathcal{V}(A_i)\), the AGI can:
\begin{enumerate}
  \item Compute combined valuations for potential outcomes using \(\land,\lor,\to\).
  \item Seek actions with high overall goodness, but acceptable badness.
  \item Update valuations over time as new evidence arrives.
\end{enumerate}

\subsection{Contradictory Evidence}

Classical logic would explode or force a retraction when faced with contradictory statements. 
Our logic permits partial overlap in the valuations and allows the system to maintain a workable 
internal state even when not all signals align perfectly. This non-explosive nature is highly 
desirable in large-scale, open-domain AGI systems subject to inconsistent data.

\section{Extensions and Future Work}

\begin{itemize}
    \item \textbf{Quantifiers:} Extending this G--B logic to a first-order setting 
    requires carefully defining the valuations of \(\forall x\) and \(\exists x\).
    \item \textbf{Dynamic Updates:} A Bayesian-like mechanism could update \((G,B)\) pairs 
    as new evidence appears, maintaining partial uncertainty and moral weighting.
    \item \textbf{Completeness Theorems:} A formal proof system (e.g., sequent calculus) 
    can be built, and completeness/soundness theorems established, for the semantics outlined here.
    \item \textbf{Practical Implementations:} Benchmark tasks that compare G--B logic 
    to classical or standard fuzzy logic would clarify the benefits and trade-offs 
    in real AGI prototypes.
\end{itemize}

\section{Conclusion}

We have introduced a \emph{good/bad} multi-valued logic that assigns every proposition two real 
values representing goodness and badness. By allowing these values to overlap, the logic 
supports partial contradictions and avoids the explosive problems of classical logic. 
We have shown how standard bivalent truth emerges as a subcase of our valuations, 
and how we might embed the system in an AGI reasoning architecture to handle complex, 
morally nuanced domains. 

This approach opens the door for more \emph{flexible} and \emph{context-aware} knowledge 
representation, particularly relevant for ethical decision-making and robust, non-explosive 
reasoning under contradictory inputs. Future work will involve extending the logic to 
quantified statements and demonstrating its practical utility in simulated or real AGI 
environments.

\section*{References}

\begin{thebibliography}{9}

\bibitem{zadeh1965fuzzy}
L.~A. Zadeh,
\newblock \emph{Fuzzy sets},
\newblock Information and Control, 8(3), 1965, pp. 338--353.

\bibitem{priest2002paraconsistent}
G.~Priest,
\newblock \emph{Paraconsistent Logic},
\newblock In Handbook of Philosophical Logic (2nd ed.), Kluwer, 2002, pp. 287--393.

\bibitem{russell2010artificial}
S.~Russell, P.~Norvig,
\newblock \emph{Artificial Intelligence: A Modern Approach} (3rd ed.),
\newblock Pearson, 2010.

\end{thebibliography}

\end{document}